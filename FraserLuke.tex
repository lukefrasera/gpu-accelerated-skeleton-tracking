\documentclass[conference]{IEEEtran}
\usepackage{cite}
\ifCLASSINFOpdf
  \usepackage[pdftex]{graphicx}
  \graphicspath{{images/}}
\else
  \usepackage[dvips]{graphicx}
  \graphicspath{{images/}}
\fi
\usepackage[cmex10]{amsmath}
\interdisplaylinepenalty=2500
\usepackage{algorithmic}
\usepackage{array}
\ifCLASSOPTIONcompsoc
 \usepackage[caption=false,font=normalsize,labelfont=sf,textfont=sf]{subfig}
\else
 \usepackage[caption=false,font=footnotesize]{subfig}
\fi
\usepackage{fixltx2e}
%\usepackage{stfloats}
%\fnbelowfloat
% \usepackage{dblfloatfix}
\usepackage{url}

\hyphenation{op-tical net-works semi-conduc-tor}

\begin{document}

\title{GPU Accelerated Skeleton Tracking From a single Depth Image}

\author{\IEEEauthorblockN{
  Luke Fraser\IEEEauthorrefmark{1},
  Monica Nicolescu\IEEEauthorrefmark{2}, 
  Freddrick Harris\IEEEauthorrefmark{3},
  Lee Barford\IEEEauthorrefmark{4}}
}

\maketitle

\begin{abstract}
Skeleton tracking is an important component of robotics. Receiving high frequency and precise pose estimates from a person are useful when a robot interacts with a person. Knowing where a person allows for more accurate real-time planning. In this paper we present a skeleton tracking algorithm that provides accurate high resolution pose estimations in real-time. This is achieved by using the GPU to take some of the load off of the CPU.
\end{abstract}

\IEEEpeerreviewmaketitle

\section{Introduction}
Skeleton tracking is a very useful tool in many fields of computer science. whether in human computer interaction, computer vision, robotics, graphics, and etc, skeleton tracking plays an important role in understanding human actions and behaviors. A lot of work has been done on the implementation of skeleton tracking \cite{Ganapathi2010,Bleiweiss2009,Baak2011,Plagemann2010,Knoop2009}. The methods can be broken down into several groups:(TODO: List groups). The most notable method is the data driven\cite{Baak2011}(TODO: cite microsoft and other data driven methods).

Data driven methods have produced robust real-time solutions to skeleton tracking\cite{Baak2011}. (TODO: Microsoft) produced a data driven method by training against a large set of a priori pose sets. The model was built from a very large data set of different people in different poses with a priori joint location information. Most data driven methods build a database of poses to compare against and then lookup the position of the current pose from the database at runtime. This allows the for low computation and quick convergence onto a solution for a given pose. However, data driven approaches rely on the a database that is complete. If the user moves into a pose that is not in the database then the system will fail.

 

In \cite{Baak2011} the authors were able to achieve 60 fps on the Swiss-Ranger 4000. 60 fps is a good rate for real-time robotics applications. This would provide the speed necessary for real-time robotic intent recognition. However, this speed was achieved at low resolutions where the number of pixels in the depth image are manageable.

\section{Related Work}
\label{sec:relatedwork}

\section{Methods}
\label{sec:method}

\section{Conclusion}
\label{sec:clonclusion}


%\begin{figure}[!t]
%\centering
%\includegraphics[width=2.5in]{myfigure}
% where an .eps filename suffix will be assumed under latex, 
% and a .pdf suffix will be assumed for pdflatex; or what has been declared
% via \DeclareGraphicsExtensions.
%\caption{Simulation results for the network.}
%\label{fig_sim}
%\end{figure}

%\begin{figure*}[!t]
%\centering
%\subfloat[Case I]{\includegraphics[width=2.5in]{box}%
%\label{fig_first_case}}
%\hfil
%\subfloat[Case II]{\includegraphics[width=2.5in]{box}%
%\label{fig_second_case}}
%\caption{Simulation results for the network.}
%\label{fig_sim}
%\end{figure*}

%\begin{table}[!t]
%% increase table row spacing, adjust to taste
%\renewcommand{\arraystretch}{1.3}
% if using array.sty, it might be a good idea to tweak the value of
% \extrarowheight as needed to properly center the text within the cells
%\caption{An Example of a Table}
%\label{table_example}
%\centering
%% Some packages, such as MDW tools, offer better commands for making tables
%% than the plain LaTeX2e tabular which is used here.
%\begin{tabular}{|c||c|}
%\hline
%One & Two\\
%\hline
%Three & Four\\
%\hline
%\end{tabular}
%\end{table}



\section{Conclusion}
The conclusion goes here.

\section*{Acknowledgment}


The authors would like to thank...





% trigger a \newpage just before the given reference
% number - used to balance the columns on the last page
% adjust value as needed - may need to be readjusted if
% the document is modified later
%\IEEEtriggeratref{8}
% The "triggered" command can be changed if desired:
%\IEEEtriggercmd{\enlargethispage{-5in}}

\bibliographystyle{IEEEtran}
% argument is your BibTeX string definitions and bibliography database(s)
\bibliography{IEEEabrv,../refs/master}

\end{document}
